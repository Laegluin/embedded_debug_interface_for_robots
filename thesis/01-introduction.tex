\chapter{Introduction}
\label{introduction}

As robots become cheaper and more popular for common tasks, they are frequently used as independent
platforms without significant supporting infrastructure or personnel. Because of this, it is
important that robots are equipped with easy to use interfaces to control them and diagnose problems.
Touchscreen displays are ideal for this use case, as they can be attached to any even surface and do
not require any specialized input devices.

This thesis implements an example of such an interface as a standalone microcontroller board
that can be directly attached to a robot built on the \textit{Wolfgang} robot platform (see section
\ref{introduction/motivation} for more details). It can be used to debug common issues with the devices
used by the robot.

Chapter \ref{related-work} presents related work in robot interface design, focusing on robots
designed for frequent interaction with humans. Chapter \ref{basics} explains some basics as well
as the bus and protocol used by the implementation. Chapter \ref{implementation} describes the
actual implementation and rationale behind important design decisions. Chapter \ref{evaluation}
provides a short overview of the usability of the finished work and relevant benchmarks while chapter
\ref{discussion} discusses these results. Chapter \ref{conclusion-and-future-work} summarizes the
findings and lists possible improvements to the current implementation.

This introduction gives a brief motivation for this bachelor thesis in section \ref{introduction/motivation}
and defines the goals in section \ref{introduction/thesis-goal}.

\section{Motivation}
\label{introduction/motivation}

% TODO: better robocup explanation?
The work done in this thesis is primarily intended for use with \textit{Wolfgang} robot platform
used by the \href{https://www.robocup.org/}{\textit{RoboCup}} team \textit{Hamburg Bit-Bots}.
It consists of various devices~\cite{bit-bots-specs} that communicate using the 
\textit{ROBOTIS Dynamixel Protocol 2.0}~\cite{dynamixel-protocol-2} over an \textit{RS-485} bus.

Without a computer connected to the robot, it is not possible to monitor the status of the connected
devices. Devices may be unreachable for different reasons:

\begin{itemize}
    \item the device is physically disconnected
    \item the device is powered off or otherwise malfunctioning
    \item the device's packets are lost, either due to interference or a misbehaving bus participant
    \item the device never sends any packet because it is waiting for another device that is not sending
          for one of the reasons above
\end{itemize}

While it is always possible to connect a computer in case of an obvious malfunction, this is a
significant amount of overhead. It does not allow for quick detection of disconnected devices
or anomalous readings of a single device.

Due to the extensible nature of the protocol, other robot platforms using the same protocol and
a bus compatible with a \textit{UART (Universal Asynchronous Receiver Transmitter)} interface will
also work, with code changes only required for adding support for new device models.

\section{Thesis Goal}
\label{introduction/thesis-goal}

The goal of this thesis is to determine a suitable microcontroller board and develop
the required firmware for it that

\begin{itemize}
    \item collects status information of the connected devices by passively listening on the bus
    \item displays this information on an integrated touchscreen display
    \item allows navigating between detailed views for each device/model using the touchscreen
    \item allows adding support for new device models with minimal effort
\end{itemize}

In particular, it should be possible to identify unreachable or disconnected devices at a glance.

Both display and microcontroller should be compact enough to be able to attach and detach them
from a robot quickly.
