\chapter{Discussion}
\label{discussion}

This chapter discusses the results gathered from the experiments that were presented and evaluated in
chapter \ref{evaluation}.
\bigbreak

All of the experiments except for \ref{evaluation/results/synthetic-ping-instructions-no-delay} and
\ref{evaluation/results/synthetic-ping-instructions-100us-delay} showed a maximum \textit{time between buffers}
of below or slightly above \SI{16}{\milli\second}. This means that in practice, the implementation is
performant enough to not lose any data, especially considering that a bus load of \SI{100}{\percent}
is not possible in reality.

It is clear that there is a constant overhead of about \SI{4}{\milli\second} in the
\textit{time between buffers}. This overhead could be seen as the median \textit{time between buffers}
in all experiments. This overhead is not small; compared to the approximate maximum of \SI{16}{\milli\second},
it is \SI{25}{\percent} of the entire time spent.

From both experiments that used synthetic \textit{Ping} instructions as data it is apparent that
\textit{Ping} instructions are the absolute worst case when it comes to processing time. This is
not surprising considering that \textit{Ping} instructions may require allocating a new
\lstinline{ControlTable} object. However, these allocations are only necessary when the model number
changes constantly as was the case in these two experiments. This is simply not a realistic scenario;
in practice, \textit{Ping} instructions are incredible rare, usually only sent when the master
connects to the bus for the first time. Thus, the number of \textit{Ping} instructions in the other
experiments was already higher than to be expected, yet it did not impact performance in any
meaningful way.

Generally, the amount of load on the bus was noticeable when interacting with the UI but this did
become a problem before incoming data started to get lost, at which point the system could not perform
its intended task anyway. Nevertheless, there certainly is room for improvement.
