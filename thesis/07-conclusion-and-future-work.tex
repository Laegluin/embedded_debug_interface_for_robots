\chapter{Conclusion and Future Work}
\label{conclusion-and-future-work}

This chapter gives a conclusion on the work done as well as the results presented in section
\ref{conclusion-and-future-work/conclusion}. Section \ref{conclusion-and-future-work/future-work}
discusses possible future work.

\section{Conclusion}
\label{conclusion-and-future-work/conclusion}

In this thesis firmware for the STM32F7508-DK board has been developed that allows monitoring and
debugging the status of devices connected to a UART compatible ROBOTIS Dynamixel Protocol 2.0 based
bus. The UI has been designed to clearly highlight any disconnected devices.

The firmware is performant enough to handle the data rate of \SI{2000000}{Bd} used with the
\textit{Wolfgang} robot platform. Future improvements would allow for even higher data rates or
multiple bus connections (see section \ref{conclusion-and-future-work/future-work}).

The source code of the firmware makes it easy to add support for new device models or update the
definition of already supported ones.

\section{Future Work}
\label{conclusion-and-future-work/future-work}

There are various ways in which the current implementation can be improved:

\begin{itemize}
    \item Currently, the device details view shows all fields in the device's \textit{control table}.
          Only a small number of values for these fields are ever observed. There is no way to
          differentiate a default value from a value that was actually observed. Default values could
          be highlighted in a different color or left out altogether.
    \item It should be possible to accept data from more than one \textit{RS-485} bus at the same time
          without difficulties. Further investigation would be required on how to physically connect
          additional buses. The increased data rate most likely also requires performance improvements.
    \item Performance may be improved by increasing the size of the receive buffers. This may also
          allow to improve UI performance by increasing the yield time of the data processing task.
    \item Performance may be improved by further optimizing the code and data structures used. This
          is unlikely to result in large improvements, however.
    \item Performance could also be improved by using more powerful hardware.
    \item Some display flickering may be removed by enabling and using V-Sync with the LCD controller.
    \item UI performance may be improved by using the dedicated 2D hardware accelerator included in
          the STM32F750N8H6 microcontroller~\cite{mcu-ref-manual}. This is unlikely to have a noticeable
          effect, as the UI task appears to spent most of the time updating the state of the widgets.
    \item UI performance may be improved significantly by using a dual core CPU. A dedicated CPU core
          could be used for rendering the UI, which would also decrease latency and simplify the
          code since the data processing task would not have to yield to allow for UI updates.
\end{itemize}
