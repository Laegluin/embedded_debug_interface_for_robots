\section*{Abstract}
\label{abstract}
\addcontentsline{toc}{chapter}{Abstract}

Modern robot platforms consist of a multitude sensors, servos and other interconnected devices.
Particularly in the case of robots that are designed to operate independently, it is often required
to diagnose problems with these devices without connecting the robot to a dedicated computer.

This bachelor thesis describes the hardware layout and software implementation for a microcontroller
board with an LCD touchscreen that can be attached directly to a robot. It continuously listens on
an \textit{RS-485} bus using the \textit{ROBOTIS Dynamixel Protocol 2.0}~\cite{dynamixel-protocol-2}
and displays detailed information about supported devices.

Other robot platforms using the same protocol would also work as long as they use a bus compatible
with a UART (Universal Asynchronous Receiver Transmitter) interface.

\begin{otherlanguage}{ngerman}
\vfill
\section*{Zusammenfassung}
\label{zusammenfassung}

Moderne Roboterplattformen bestehen aus einer Vielzahl von Sensoren, Servomotoren und anderen
miteinander verbundenen Geräten. Insbesondere im Fall von Robotern, die darauf ausgelegt sind
unabhängig zu agieren, ist es oft notwendig Probleme mit diesen Geräten zu diagnostizieren ohne den
Roboter an einen dedizierten Computer anzuschließen.

Diese Bachelorarbeit beschreibt den Hardware-Aufbau und die Software"=Implementation für eine
Mikrocontrollerplatine mit einem LCD Touchscreen, die direkt an einen Roboter montiert werden kann.
Diese hört kontinuierlich einen \textit{RS-485} Bus ab, der das \textit{ROBOTIS Dynamixel Protocol 2.0}~\
\cite{dynamixel-protocol-2} verwendet, und zeigt detailierte Informationen über unterstützte Geräte an.

Andere Roboterplattformen, die das gleiche Protokoll verwenden, würden auch funktionieren, solange
sie einen Bus benutzen, der mit einer UART (Universal Asynchronous Receiver Transmitter) Schnittstelle
kompatibel ist.

\end{otherlanguage}
